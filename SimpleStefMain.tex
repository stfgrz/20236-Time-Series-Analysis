%Hey, if you're using this preamble it means that it was probably written by Stefano Graziosi (me). If you see something that doesn't make sense, feel free to email me at stefano.graziosi@studbocconi.it
%p.s. in case it's not already evident from the preamble, I'm not a professional LaTeX user, so I'm sure there are better ways to do things. I'm just trying to make it work.

%------------------------------------------------------------------------------
%           LAST UPDATE: 06-01-2025
%------------------------------------------------------------------------------

%I don't own copyright on anything, I just literally copied and pasted together a bunch of stuff.

%Credit goes to the original authors.

%------------------------------------------------------------------------------
%           Packages
%------------------------------------------------------------------------------

\usepackage[all]{xy}
\usepackage{tcolorbox}
\usepackage{graphicx}
\usepackage{hyperref}
\usepackage{xcolor}    
\usepackage{wrapfig}
\usepackage{amsmath, amssymb, amsthm}
\usepackage{titlesec}
\usepackage{halloweenmath}
\usepackage{enumitem}
\usepackage{listings}

\usepackage{titlepic}
\usepackage{pdfpages}

%------------------------------------------------------------------------------
%           Geometry
%------------------------------------------------------------------------------

\usepackage[a4paper,margin=1in]{geometry}
%\usepackage[margin=1in]{geometry}


%------------------------------------------------------------------------------
%           Colours
%------------------------------------------------------------------------------



%------------------------------------------------------------------------------
%           Environments
%------------------------------------------------------------------------------

%Standard \latex box

\newtcolorbox{mybox}[3][]
{
  colframe = #2!25,
  colback  = #2!10,
  coltitle = #2!20!black,  
  title    = {#3},
  #1,
}

%Standard "Problem" environment

\newtheorem{problem}{Problem}

%Personalised "Solution" environment

\newenvironment{solution}[1][\it{Solution}]{\textbf{#1. } }{$\square$}

%------------------------------------------------------------------------------
%           Code Listing Environment
%------------------------------------------------------------------------------

\lstdefinestyle{mystyle}{
    backgroundcolor=\color{backcolour},   
    commentstyle=\color{codegreen},
    keywordstyle=\color{magenta},
    numberstyle=\tiny\color{codegray},
    stringstyle=\color{codepurple},
    basicstyle=\ttfamily\footnotesize,
    breakatwhitespace=false,         
    breaklines=true,                 
    captionpos=b,                    
    keepspaces=true,                 
    numbers=left,                    
    numbersep=5pt,                  
    showspaces=false,                
    showstringspaces=false,
    showtabs=false,                  
    tabsize=2
}

\lstset{style=mystyle}